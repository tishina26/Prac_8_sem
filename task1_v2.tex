\documentclass{article}


\usepackage[T1, T2A]{fontenc}
\usepackage[russian]{babel}
\usepackage{amsmath}
\usepackage{graphicx}
\usepackage{color}
\usepackage{float}


\usepackage{hyperref} % Для создания гиперссылок


\makeatletter
\newcommand{\@begintheorem}[2]{\begin{trivlist}\it
  \itemindent=\parindent
  \item[\hspace{\labelsep}{\bf #1\ #2.}]}
\newcommand{\@endtheorem}{\end{trivlist}}

\newtheorem{testcase}{Тест}%[section]

\usepackage{listings}
\usepackage{caption}
%\DeclareCaptionFont{white}{\color{white}} % Для белого текста, если нужно
\DeclareCaptionFormat{listing}{\colorbox{gray}{\parbox{\dimexpr\linewidth-2\fboxsep}{\centering #1#2#3}}}
\captionsetup[lstlisting]{format=listing, singlelinecheck=false, margin=0pt, font={bf,footnotesize}}

\lstset{
    language=Python,
    basicstyle=\ttfamily\normalsize, % Увеличенный размер шрифта
    breaklines=true,
    captionpos=b,
    %frame=single, % Рамка вокруг кода (опционально)
    numbers=left, % Номера строк (опционально)
    numberstyle=\tiny,
}




\usepackage{algorithm}
\floatname{algorithm}{Алгоритм} % Изменяем название "Algorithm" на "Алгоритм"
\newcommand{\algorithmicrequire}{\textbf{Входные данные:}}
\newcommand{\algorithmicensure}{\textbf{Выходные данные:}}
\newcommand{\algorithmicwhile}{\textbf{пока} }
\newcommand{\algorithmicEndWhile}{\textbf{Конец цикла}}
\newcommand{\algorithmicend}{\textbf{Конец цикла}} % Ключевое изменение
\newcommand{\algorithmicdo}{\textbf{выполняй}} % Ключевое изменение
\newcommand{\algorithmicfor}{\textbf{для}}
\newcommand{\algorithmicendfor}{\textbf{Конец цикла}} % И здесь тоже
\newcommand{\algorithmicreturn}{\textbf{Вернуть} }
\usepackage{algpseudocode}

\title{Отчет по решению задачи оптимизации методом Франка-Вульфа}
\author{Тишина Ульяна}
\date{\today} % Можно указать конкретную дату, если нужно

\usepackage[russian]{datetime2}
%\date{18 марта 2025 г.} % Или любой другой нужный формат


\begin{document}

%\selectlanguage{russian} % Выбираем русский язык перед \maketitle
\maketitle

\renewcommand{\contentsname}{Содержание} % Переопределяем название содержания

\newpage
\tableofcontents


\newpage
\section{Постановка задачи}
  $ $
  
% \addcontentsline{toc}{section}{Постановка задачи}
Рассмотрим задачу оптимизации:
\begin{equation}
f(x)  \rightarrow min,
\end{equation}
где f(x) - нелинейная функция нескольких переменных,

с линейными ограничениями:
\begin{equation}
\begin{array}{c}
A * x \ge b,\\
A\_eq * x = b\_eq
\end{array}
\end{equation}

\section{Метод Франка-Вульфа}
$ $

Метод Франка - Вульфа применяется для задач с нелинейной целевой функцией и линейными ограничениями. Метод основан на разложении нелинейной функции общего вида в ряд Тейлора до членов первого порядка в окрестности некоторой точки.


\begin{algorithm}
\caption{Метод Франка-Вульфа}
\begin{algorithmic}[1]
\Require $f(x)$, $x^0$, ограничения, критерии выхода
\State $k \gets 0$
\While{Критерии выхода не выполнены}
    \State вычислить градиент: $\nabla f(x^k) = \left(\frac{\partial f(x^k)}{\partial x_1}, \dots, \frac{\partial f(x^k)}{\partial x_n}\right)$;
    \State составить линейную целевую функцию: $F_k(y) = \nabla f(x^k) \cdot y \rightarrow \min$;
    \State решить ЗЛП с ограничениями (например, симплекс-методом), получить $y^k$;
    \State найти минимум функции $f(x^{k+1}) \rightarrow min,$
где $x^{k+1} = x^k + h^k (y^k - x^k)$, 
$h^k \in (0, 1),$
применив, например, метод золотого сечения;
    \State $x^{k+1} \gets x^k + h^k (y^k - x^k)$;
    \State $k \gets k + 1$.
\EndWhile
\State \Return $x^k$
\end{algorithmic}
\end{algorithm}






\newpage
\section{Проверка корректности реализации алгоритма}
$  $

Для проверки работы алгоритма будем использовать библиотеку scipy.optimize. В ней есть функция minimize, которая находит решение задачи оптимизации с нелинейной или линейной целевой функцией с нелинейными и/или линейными ограничениями-равенствами и/или ограничениями-неравенствами.

Сравним результат работы реализованного алгоритма Франка-Вульфа и функции минимазации из библиотеки: найденные значения целевой функции, количество итераций и статус завершения.

В реализации алгоритма Франка-Вульфа успешному завершению соответствует статус 0, не успешному статус -1. В библиотечной функции успеху также соответствует статус 0, а неуспешному - другое число, значение которого можно найти в документации библиотеки.

В таблицах представлены результаты двух алгоритмов и последний столбец - сравнение этих результатов следующим  образом: 
\begin{itemize}
\item
разница значений целевой функции (должна быть меньше заданной точности);
\item
сравнение кол-ва итераций: у какого метода меньше, тот метод лучше (будем писать для метода Франка-Вульфа (Ф-В): Ф-В лучше или хуже);
\item
сравнение статусов алгоритмов: если оба алгоритма завершились успехом, то запишем 0, иначе -1.
\end{itemize}



% Пример 1
\begin{testcase} \label{TestCase1} \em
Рассмотрим задачу оптимизации \cite[пример 1]

\begin{equation}
  \begin{array}{c}
  f(x) = x_1^{0.25} + (\frac{x_2}{x_1})^{0.25} + (\frac{64}{x_2})^{0.25} \rightarrow min,\\
   x_1 \ge 1, \\
   x_2-x_1 \ge 0, \\
   -x_2 \ge -64.
  \end{array}
\end{equation}

\begin{table}[h!]
    \begin{center}
    \begin{tabular}{|c|c|c|c|}
        \hline
         & Метод Франка-Вульфа & Минимизация & Сравнение \\
        \hline
        F & 4.242641 & 4.242641 & 0 \\
        \hline
        x & [3.999938, 15.999251] & [3.999027, 16.004494] & - \\
        \hline
        Количество итераций & 13 & 18 & Ф-В лучше \\
        \hline
        Статус & 0 & 0 & 0 \\
        \hline
    \end{tabular}
    \caption{\ref{TestCase1}, $x_0~=~(2,~10)$, \varepsilon~=~0.001.}
    \end{center}
\end{table}

\end{testcase}

\newpage

% Пример 2
\begin{testcase} \label{TestCase2} \em
Рассмотрим задачу оптимизации \cite[пример 2]

\begin{equation}
  \begin{array}{c}
  f(x) = (x_1-1)^2 + 2(x_2-1)^2-3 \rightarrow min,\\
   -x_1-x_2 \ge -8, \\
   -2 x_1+x_2 \ge -12, \\
   x_1 \ge 0, \\
   x_2 \ge 0.
  \end{array}
\end{equation}

\begin{table}[h!]
    \begin{center}
    \begin{tabular}{|c|c|c|c|}
        \hline
         & Метод Франка-Вульфа & Минимизация & Сравнение \\
        \hline
        F & -3.0 & -3.0 & 0 \\
        \hline
        x & [0.999987, 0.999818] & [1.000000, 1.000000] & - \\
        \hline
        Количество итераций & 5 & 3 & Ф-В хуже \\
        \hline
        Статус & 0 & 0 & 0 \\
        \hline
    \end{tabular}
    \caption{\ref{TestCase2}, $x_0~=~(0,~0)$, \varepsilon~=~0.001.}
    \end{center}
\end{table}

\end{testcase}



% Пример 3
\begin{testcase} \label{TestCase3} \em
Рассмотрим задачу оптимизации \cite[пример 3]

\begin{equation}
  \begin{array}{c}
  f(x) = -(-x_1^2 + x_1x_2 - 2x_2^2 + 4x_1 + 6x_2) \rightarrow min,\\
   -x_1 - x_2 \ge -4, \\
   x_1 + 2x_2 \ge 2, \\
   x_1 \ge 0,\\
   x_2 \ge 0. \\
  \end{array}
\end{equation}

\begin{table}[h!]
    \begin{center}
    \begin{tabular}{|c|c|c|c|}
        \hline
         & Метод Франка-Вульфа & Минимизация & Сравнение \\
        \hline
        F & -12.25 & -12.25 & 0 \\
        \hline
        x & [2.250000, 1.750000] & [2.250000, 1.750000] & - \\
        \hline
        Количество итераций & 1 & 2 & Ф-В лучше \\
        \hline
        Статус & 0 & 0 & 0 \\
        \hline
    \end{tabular}
    \caption{\ref{TestCase3}, $x_0~=~(3,~1)$, \varepsilon~=~0.001.}
    \end{center}
\end{table}

\end{testcase}

\newpage

% Пример 4
\begin{testcase} \label{TestCase4} \em
Рассмотрим задачу оптимизации \cite[пример 4]

\begin{equation}
  \begin{array}{c}
  f(x) = sin(x_1) + cos(x_2^2)\rightarrow min,\\
   -x_1 - x_2 \ge -4, \\
   2x_1+5x_2 \ge 1, \\
   x_1 \ge 0,\\
   x_2 \ge 0.\\
  \end{array}
\end{equation}

\begin{table}[h!]
    \begin{center}
    \begin{tabular}{|c|c|c|c|}
        \hline
         & Метод Франка-Вульфа & Минимизация & Сравнение \\
        \hline
        F & -0.999992 & -1.0 &  0.000008\\
        \hline
        x & [0.000008, 1.772453] & [0.000000, 3.963327] & - \\
        \hline
        Количество итераций & 14 & 6 & Ф-В хуже \\
        \hline
        Статус & 0 & 0 & 0 \\
        \hline
    \end{tabular}
    \caption{\ref{TestCase4}, $x_0~=~(0.1,~0.1)$, \varepsilon~=~0.0001.}
    \end{center}
\end{table}

\end{testcase}

Во всех примерах значение функции найдено успешно, алгоритм работает корректно, с заданной точностью.

\newpage

\section*{Список литературы}
\addcontentsline{toc}{section}{Список литературы}


\begin{enumerate}
    \item Лобанов В.С. Метод линеаризации для задач условной оптимизации. Алгоритм Франка-Вульфа. \url{https://moluch.ru/archive/293/66414/}
    \item Алгоритм Франка-Вульфа. (пример1) \url{https://math.semestr.ru/optim/frank.php}
    \item Алгоритм Франка-Вульфа. (пример2)  \url{https://studfile.net/preview/7775095/page:30/}
    \item Алгоритм Франка-Вульфа. (пример3) \url{https://www.matburo.ru/ex_mp.php?p1=mpnp}
\end{enumerate}

\newpage
\section*{Приложение}
\addcontentsline{toc}{section}{Приложение}%
Реализация метода Франка-Вульфа:
\begin{lstlisting}
def frank_wolf(fi, x0, A, b, A_eq, b_eq, max_iter, eps, view=0):
   xk = x0
   k = 0
   opt = {'k': 0,
          'x': x0,
          'f': 0,
          'status': -1}

   while (k <= max_iter):
      df = my_grad(fi, xk)
      if view: print(f'grad(x{k}) = {df}')

      # minimize
      if A is None:
          res = linprog(df, A_eq=A_eq, b_eq=b_eq)
      else:
          res = linprog(df, A_ub=-A, b_ub=-b, A_eq=A_eq, b_eq=b_eq)
      yk = res.x
      if yk is None:
          opt['status'] = -1
          if view: print("yk is None: ", res)
          return opt
      if res.status != 0:
          opt['status'] = -1
          if view: print("status of linprog is not 0: ", res)
          return opt
      if view: print(f'y{k} = {yk}')

      # alpha search for xk+1 = xk + ak(yk-xk) for 0<=ak<=1
      prev = xk
      xk = find_xk(fi,xk,yk)
      if view: print(f'x{k} = {xk}')

      # fill opt
      opt['k'] = k
      opt['f'] = fi(xk)
      opt['x'] = xk
      k += 1
      # break
      if (norm(xk - prev) < eps) and (abs(fi(xk) - fi(prev)) < eps):
          opt['status'] = 0
          break
   return opt
\end{lstlisting}






\newpage
Реализация функции, вычисляющей производную
\begin{lstlisting}
def my_grad(f, x, h=1e-6):
    grad = np.zeros_like(x)
    for i in range(len(x)):
    	x_plus_h = np.copy(x)
    	x_plus_h[i] += h
    	grad[i] = (f(x_plus_h) - f(x)) / h
    return grad
\end{lstlisting}


Реализация функции золого сечения


\begin{lstlisting}
def find_xk(f, a,b, tol=1e-6, max_iter = 100):
    hk = 2 - (1 + 5**0.5) / 2

    x1 = a + hk * (b - a)
    x2 = b - hk * (b - a)
    # print(x1,x2)
    f1 = f(x1)
    f2 = f(x2)
    # print(f1,f2)

    for _ in range(max_iter):
        # print(x1,x2)
        if norm(b - a) < tol:
            break
        # print("f1f2",f1,f2)
        if f1 < f2:
            b, x2, f2 = x2, x1, f1
            x1 = a + hk * (b - a)
            f1 = f(x1)
        else:
            a, x1, f1 = x1, x2, f2
            x2 = b - hk * (b - a)
            f2 = f(x2)

    return (a + b) / 2
\end{lstlisting}


\newpage

Реализация функции сравнения результатов

\begin{lstlisting}
def compare_results_df(res_fw, res_min, eps=1e-3, r=4):
    for i in range(len(res_fw)):
        fw = res_fw[i]
        mn = res_min[i]

        fun_comparison = abs(fw['f'] - mn.fun) < eps

        status_comparison = 0 if ((fw['status'] == mn.status) == 0) else -1

        x_comparison = "-"

        if fw['k'] == mn.nit:
            nit_comparison = "eq"
        elif fw['k'] > mn.nit:
            nit_comparison = "F-W worse"
        else:
            nit_comparison = "F-W better"

        data = {
            "frank_wolf": [fw['f'], fw['status'], round_arr(fw['x']), fw['k']],
            "minimize": [mn.fun, mn.status, round_arr(mn.x), mn.nit],
            "delta": [fun_comparison, status_comparison, x_comparison, nit_comparison]
        }

        df = pd.DataFrame(data, index=["f", "status", "x", "nit"])

        print(f"\nПример {i+1}:")
        print(df)
\end{lstlisting}







\end{document}
